\documentclass[]{article}

\usepackage{graphicx}
\usepackage{subcaption}


%opening
\title{Report on CNN training}
\author{Ricardo Núñez Prieto \\
		Pablo Correa Gómez}
\makeindex

\begin{document}

\maketitle


\begin{abstract}
	Explanation of progress with neural network

\end{abstract}



\section{Training in our CPUs}
\subsection{Before data augmentation}
We worked on several trainings to get to understand the different parameters and optimize them for our net.
\newline
We had 1500 images.
	\begin{figure}[h]
		\centering
		\includegraphics[width=1\textwidth]{cpu_bef_aug.png}
		\caption{Example of training before data augmentation}
	\end{figure}

\newpage
\subsection{Data augmentation}
Proceed to data augmentation. As it had been done with a script, we proceeded to do some cleaning. We compared similar images in smaller groups: AOG, MN, UVW and then removed the conflicting images that were out of the frame, a bit blurry, etc.
	\begin{figure}[h]
		\centering
		\begin{subfigure}[b]{0.49\textwidth}
			\includegraphics[width=\textwidth]{ILRTXZ_before.png}
			\caption{Before the cleaning}
		\end{subfigure}
		\begin{subfigure}[b]{0.5\textwidth}
		\includegraphics[width=\textwidth]{ILRTXZ.png}
		\caption{After the cleaning}
		\end{subfigure}
	
	\caption{ILRTXZ trained together}
	\end{figure}
\newline
After the cleaning procedure we had around 6500 images.

\paragraph*{}
Then we trained all the images together and again spent some time hand-tuning the learning parameters.

	\begin{figure}[h]
		\centering
		\includegraphics[width=1\textwidth]{cpu_pablo_all.png}
		\caption{Nice CPU training on our computers}
	\end{figure}


\section{GPU arrives}

\subsection{Training augmented dataset}
We had some mistake or something wasn't working that we can still not figure out which made the trainings useless. On Saturday Ricardo tried again and the problem was fixed.
\newline
We have done two kind of trainings: 
\begin{itemize}
	\item Splitting the dataset randomly and using 85\% of the images for training and 15\% for validation.
	\item Taking all the dataset for training and validating with some other images that the Neural Network has never seen before.
\end{itemize}

\begin{figure}[h]
	\centering
	\begin{subfigure}[b]{0.49\textwidth}
		\includegraphics[width=\textwidth]{gpu_alldataset_partitioned.png}
		\caption{Random partitioning}
	\end{subfigure}
	\begin{subfigure}[b]{0.5\textwidth}
		\includegraphics[width=\textwidth]{gpu_ric_y_esposa_validation.png}
		\caption{Validating never seen images}
	\end{subfigure}
	
	\caption{Trainings with GPU}
\end{figure}

\subsection{Training original dataset}
We tried to train the original dataset again to check that the data augmentation has really made a difference.

\begin{figure}[h]
	\centering
	\begin{subfigure}[b]{0.49\textwidth}
		\includegraphics[width=\textwidth]{gpu_before_aug.png}
		\caption{Random partitioning}
	\end{subfigure}
	\begin{subfigure}[b]{0.5\textwidth}
		\includegraphics[width=\textwidth]{gpu_before_aug_ric__validation.png}
		\caption{Validating never seen images}
	\end{subfigure}
	
	\caption{Trainings with GPU original dataset}
\end{figure}

\subsection{Training with googlenet}

As suggested by Liang, we have tried training with a different net to check if the problem identified is due to our topology. After training with googlenet we can conclude that it is not.

\begin{figure}[h]
	\centering
	\begin{subfigure}[b]{0.49\textwidth}
		\includegraphics[width=\textwidth]{google_regular_training.png}
		\caption{Random partitioning}
	\end{subfigure}
	\begin{subfigure}[b]{0.5\textwidth}
		\includegraphics[width=\textwidth]{google_ric_validation.png}
		\caption{Validating never seen images}
	\end{subfigure}
	
	\caption{Trainings original dataset}
\end{figure}



\end{document}
